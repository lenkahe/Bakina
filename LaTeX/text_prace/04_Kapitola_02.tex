\chapter{Matematická teória}

V tejto kapitole uvedieme potrebné znalosti z matematickej teórie, ktoré sú potrebné pre zostrojenie optimalizačných algoritmov, ktorými sa budeme zaoberať v následujúcich kapitolách. Budú použité definície a pojmy zo zdrojov ...
%Po dôkladnom oboznámení sa s dátami, ktoré generuje performačný nástroj Perfcake sme usúdili, že ak dáta a teda grafy %``vyhladíme'', užívateľ nestratí podstatnú informáciu, ktorú dáta nesú. Po aplikovaní kĺzavých priemerov, sme sa %rozhodli použiť zovšeobecnenie tejto metódy a teda metódu jadrových odhadov. Prvá časť tejto kapitoly sa venuje úvodu %do tejto metódy. Budú použité definície a pojmy zo zdroja 1 a 2. 

%Po aplikovaní 
\section{Regresná analýza}
Pre pevné hodnoty nezávisle premennej $X$ (v našom prípade čas)  máme k dispozícií namerané hodnoty závisle premennej $Y$ (priepustnosť, použitá pamäť,...). Takýmito dvojicami bodov $(x_i, Y_i), i = 1,...,n$ chceme preložiť vhodnú krivku, tak aby boli odfiltrované výkyvy a bolo možné lepšie poznať štruktúru dát. Táto krivka sa nazýva \textit{regresná krivka} a jej príslušný regresný vzťah zapisujem v tvare

\begin{equation}
Y_i = m(x_i) + \varepsilon _i, i=1,...,n,
\end{equation}

kde \textit{m} je neznáma regresná funkcia a $\varepsilon _i, i = 1,...,n$, sú chyby merania. Cieľom regresnej analýzy je nájsť vhodnú aproximáciu $\hat{m}$ neznámej funkcie $m$. Hľadanie tejto regresnej krivky sa tiež nazýva \textit{vyhladzovanie} a je možné použiť dva spôsoby odhadu, \textit{parametricky} a \textit{neparametricky}:
\begin{itemize}
\item \textit{Parametrický prístup} - predpokladá, že regresná funkcia je nejakého predpísaného tvaru. Odhadnutá regresná funkcia bude teda určitého tvaru a bude ju popisovať množina parametrov - to je dôvod pre názov \textit{parametrický}.
\item \textit{Neparametrický prístup} - nepredpokladá predpísaný tvar regresnej funkcie. Tento prístup sa vyhýba parametrizácii a tvar funkcie sa odhaduje priamo z dát. Predpokladá sa jedine istá hladkosť hľadanej funkcie.
\end{itemize}

Jedna z najjednoduchších neparametrických metód je \textit{ metóda kĺzavých priemerov}. Pre odhad hodnoty $Y_i$ sa používa priemer niekoľkých hodnôt $Y_j,  j\in [i-h,i+h]$ v centrovanom okolí príslušného bodu $x_i$. \\*
Konkrétne,
 
\begin{equation}
\hat{m}(x) = \dfrac{\sum\limits_{i=1}^{n}  Y_i I (x_i  \in  [x - h, x + h])}{\sum\limits_{i=1}^{n} I(x_i  \in  [x - h, x + h])}.
\end{equation}
Jadrové odhady sa považujú za zovšeobecnenie \textit{metódy kĺzavých priemerov}.


\section{Jadrové odhady}

Pri odhadovaní regresnej funkcie pomocou jadrových odhadov, sa taktiež používajú vážené hodnoty $Y$ v centrovanom okolí príslušného bodu $x_i$. Váhy hodnôt $Y$ sú závislé na vzdialenosti príslušných $x$ bodov od konkrétneho $x_i$, bližšie hodnoty  -  väčšia váha. Toto nám pomáha dosiahnuť \textit{jadrová funkcia}. Okrem jadrovej funkcie, alebo jadra, je ďalším dôležitým parametrom tejto metódy je šírka vyhladzovacieho okna $h$.
Vzorec pre jadrové odhady vo všeobecnosti, môžme zapísať nasledovne
\begin{equation}
\hat{m}(x,h) = \frac{1}{n}\sum\limits_{i=1}^{n} K_h(x)Y_i,
\end{equation}
kde $K_h(x)$ je váhová funkcia s vyhladzovacím parametrom \textit{h} a môže mať tvar
\begin{equation}
K_h(x - x_i) = \frac{1}{h}K\Big(\frac{x-x_i}{h}\Big),
\end{equation}
kde \textit{K} je jadrová funkcia.

\subsection{Jadrová funkcia}
\begin{figure}[!ht]
\centering
\begin{subfigure}{.3\textwidth}
  \centering
  \includegraphics[scale=0.15]{Epanecnikovo.pdf}
  \caption{Epanečnikovo jadro}
\end{subfigure}%
\begin{subfigure}{.3\textwidth}
  \centering
  \includegraphics[scale=0.15]{quadratic.pdf}
  \caption{Kvadratické jadro}
\end{subfigure}
\begin{subfigure}{.3\textwidth}
  \centering
  \includegraphics[scale=0.15]{Gaussian.pdf}
  \caption{Gaussovo jadro}
\end{subfigure}
\begin{subfigure}{.3\textwidth}
  \centering
  \includegraphics[scale=0.15]{uniform.pdf}
  \caption{Obdĺžnikové jadro}
\end{subfigure}
\begin{subfigure}{.3\textwidth}
  \centering
  \includegraphics[scale=0.15]{triangular.pdf}
  \caption{Trojuholníkové jadro}
\end{subfigure}
\begin{subfigure}{.3\textwidth}
  \centering
  \includegraphics[scale=0.15]{smiesne.pdf}
  \caption{Nepomenované jadro}
\end{subfigure}
\centering
\caption{Rôzne tvary jadrových funkcií:}\label{kernelShapeTypes}
\begin{tabular}{ l l l }
 a) $K(x)=\frac{3}{4}(1-x^2)I_{[-1,1]}$, & b) $K(x)=\frac{15}{16}(1-x^2)^2I_{[-1,1]}$, & c) $K(x)=\frac{1}{\sqrt{2\pi}}e^-\frac{x^2}{2} $, \\
 d) $K(x)=\frac{1}{2}I_{[-1,1]} $, & e) $K(x)=(1-|x|)I_{[-1,1]}$, & f) $K(x)=\frac{1}{2}e^{-|x|}$
 \end{tabular}
 
\end{figure}

Jadrová funkcia determinuje tvar vyhladzovacej funkcie. Na obrázku \ref{kernelShapeTypes} môžme vidieť niekoľko najpoužívanejších jadrových funkcií. Na vyjadrenie 
\begin{equation}
I_{[-1,1]}(x) \begin{cases}
1 & x \in [-1,1], \\ 
0 & $inak.$
\end{cases}
\end{equation}
sa používa \textit{indikátorová funkcia} $I_{[-1,1]}(x)$.

Vo všeobecnosti hocijaká integrovatelná, obmedzená funkcia, ktorá spĺňa kritériá \ref{criterion} môže byť jadrom.
\begin{align}\label{criterion}
\begin{split}
a) \int K(z)dz = 1 \qquad & b) \int zK(z)dz = 0 \\ 
c) \int z^2K(z)dz < \infty \qquad & d) K(x) \geq 0 \enspace \textrm{pre všetky} \enspace x.
\end{split}
\end{align} 

\subsection{Šírka vyhladzovacieho okna}

Šírka vyhladzovacieho okna alebo aj vyhladzovací parameter \textit{h} udáva šírku vyhladzovacej funkcie a teda aj silu vyhladenia. 

Malá šírka vyhladzovacieho okna znamená, že odhad závisí na úzkom okolí bodu $x_i$ a teda odhad do veľkej miery reprodukuje pôvodné dáta. Naopak ak zvolíme vysokú hodnota $h$, aj veľmi vzdialené hodnoty majú vysoký dopad na odhad, čo vedie k ``prehladeniu'' a pri dostatočne veľkej šírke $h$ až k priemeru dát. Spomínané rozdieli v šírke vyhladzovacieho okna môžme vidieť na obrázku \ref{porovnanieSirky}. Pre ilustráciu vplyvu vyhladzovacieho parametra sú použite Nadaraya-Watsonove odhady, ktoré sú popísané v nasledujúcej podkapitole. 
 
 \begin{figure}[!ht]
  \includegraphics[scale=0.25]{kernelEstimationMemory005.pdf}
  \includegraphics[scale=0.25]{kernelEstimationMemory8.pdf}
  \centering
  \includegraphics[scale=0.25]{kernelEstimationMemory2.pdf}
  \caption{Porovnanie šírky vyhladzovacieho okna, bodky sú pôvodné dáta, súvislá čiara odhadnutá regresná funkcia}\label{porovnanieSirky}
\end{figure}
 
\subsection{Typy jadrových odhadov}
V tejto podkapitole si uvedieme konkrétne typy jadrových odhadov, ktoré sú uvedené v zdrojoch ...

Medzi najznámejšie jadrové odhady patria:

\begin{itemize}
\item  \textbf{Nadaraya-Watsonove odhady}

\begin{equation*}
\hat{m}_{MW}(x;h) = \frac{\sum\limits_{i=1}^{n} K_h(x - x_i)Y_i}{\sum\limits_{i=1}^{n} K_h(x - x_i)},
\end{equation*}

\item
\textbf{Gasser-Müllerove odhady}
\begin{equation*}
\hat{m}_{GM}(x;h) = \sum\limits_{i=1}^{n} Y_i \int\limits_{s_i-1}^{s_i}K_h(t-x)dt,
\end{equation*}

kde $s_0 = 0 , s_i = \frac{x_i + x_{i+1}}{2}, s_n = 1$.

\item
 \textbf{Priestley-Chaove odhady}
\begin{equation*}
\hat{m}_{PCH}(x;h) = \frac{1}{n}\sum\limits_{i=1}^{n}K_h(x-x_i)Y_i
\end{equation*}
\end{itemize}

Pre odhad regresnej funkcie je možné použiť rovnako odhady $\hat{m}_{NW}, \hat{m}_{} $ 